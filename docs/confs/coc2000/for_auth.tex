\documentstyle{coc2000}
\begin{document}
\bibliographystyle{unsrt}
\pagestyle{empty}
\newtheorem{lemma}{Lemma}

\title{Author's Guide\\
2nd International Conference\\
``CONTROL OF OSCILLATIONS AND CHAOS" (COC'2000)
}

\author{Alexander N. Churilov\\
Dept. of Computer Science,
Marine Technical University\\
3 Lotsmanskaya Str.,
St.Petersburg 190008, Russia\\
churil@gelig.usr.pu.ru}
\date{}
\maketitle

\begin{abstract}
This document provides instructions for
preparing a paper for Proceedings of the
2nd International Conference
``Control of Oscillations and Chaos" (COC'2000).
Two volumes of proceedings are planned which
will be available on-site by the registration
at the beginning of the conference. Since the
proceedings will be printed by offset from
the original prepared by you, it is imperative
that sizes of the text box, the margins,
and style described in this
guide be adhered to carefully. Our guidelines
basically follow those adapted for IEEE CDC
publications. In particular, we used the instruction
written by C.B.~Shrader for 35th IEEE Conference on
Decision and Control (December, 1996).
The guidelines detailed in this document will
enable the Conference to maintain uniformity in
the proceedings and to avoid additional
publication charges.
This document describes the preparation of exact
size manuscripts. To ensure high-quality
proceedings readability of your original manuscript
is of paramount importance.
\end{abstract}

\section{Manuscript Submission}

Authors submit their manuscripts to the
Organizing Committee.
{\bf Two copies} of the camera-ready original of
your paper must be sent by post, not fax, at the
address given below. Your manuscript must
be received by {\bf June 15, 1999}.
The address is:
\begin{center}
Prof.A.L.Fradkov\\
The Institute for Problems of Mechanical Engineering\\
61 Bolshoy ave. V.O.\\
199178, St.Petersburg, RUSSIA\\
Tel: +7(812)321-4766, Fax: +7(812)321-4771\\
E-mail: coc2000@ccs.ipme.ru,
http://www.ipme.ru/coc2000.html
\end{center}
Please, apply exclusively the address pointed above, and
do not send your manuscripts to the members
of Program or Organizing Committees.

\fbox{\bf
\parbox{75mm}{In no circumstances should
your paper or even any part of it be sent in
electronic form.}
}

International participants should take into
account that the state post service is rather
slow in Russia (e.g., it takes about month
to deliver a letter from USA to St.Petersburg),
so commercial post services are recommended.

\section{General Specifications}

\subsection{Important Information}

\begin{enumerate}
\item You need to submit a camera-ready manuscript
using a laser printer, 300 dpi or more,
on exact size A4 (210~mm by 297~mm) white paper, using one side
of a sheet.

\item  Those participants who will deliver a Plenary
or Semi-Plenary Lecture should submit a full-size
paper of six (6) pages, each Regular Lecture is allotted
a paper of four (4) pages in the proceedings.
Each manuscript must follow this rule.  Authors with
multiple submissions must apply these guidelines to each
manuscript separately.
Up to three additional pages will be permitted for
a charge of 80 US Dollars for each additional page.

\item The deadline for receipt of your camera-ready
manuscript by the Organizing Committee is
{\bf April 15, 2000}.
Papers received after the deadline, may not be
included in the proceedings.

\item The title and authorship of your paper will
appear in the final program, in the table of contents,
and author index of the proceedings as they are
shown in your manuscript submitted to
the Organizing Committee.

\item When mailing your manuscript to the
Organizing Committee, make sure to include your
complete mailing address and type of your report
(Plenary, Regular or Poster)
on the {\bf outside} of the envelope.
\end{enumerate}

\subsection{Preparation of Exact Size Manuscripts}

Your paper must be printed actual size (i.e., exactly
how it is to appear in the proceedings) in two columns.
To ensure uniformity of appearance for the proceedings,
the papers should conform to the following specifications.
If your paper deviates significantly from these
specifications, the printer may not be able
to include your paper in the proceedings.
It is highly recommended to use \LaTeX~ in preparing and
formatting your manuscript (see below).

The hight of the text block should be {\bf 240~mm}.
The width of each column should be {\bf 80~mm}.
The distance between the two columns of text
should be {\bf 5~mm}. Thus, the width of the text block
is {\bf 165~mm}.
The text should be centered left-to-right
on the page, i.e., left and right margins
should be the same.
The distance from the bottom
edge of the paper to the bottom of the last
line of type on the page should be not less than {\bf 30~mm}.
This allows room for the printer to
insert the copyright notice on the first page
and page numbers.

Please, do not insert page numbers or other informaton into
running headers or footers. This will be done by us.

The authors of COC'97 should take into account that
the page layout described above differs from that was
used for COC'97 Proceedings.

\subsection{Required Font Sizes}

For the main body of the text the {\bf 10} point size roman font
should be used.
Some technical formatting programs print
mathematical formulas in italic type, with subscripts and
superscripts in a slightly smaller font size. This is acceptable.

\subsection{Title} The title should be centered across the top
of the first page. The {\bf 22} point size bold font is recommended.

\subsection{Authors' Names and Addresses} The authors' names and
addresses should be centered below the title. The {\bf 14} point
size roman font is recommended. Please include your e-mail address.
For multiauthored papers names and addresses can be placed
either side by side, or one under another. Please do not
put addresses in footnotes.

\subsection{Paragraphs}
Do not indent the initial lines of paragraphs.
Leave a line clear between paragraphs.

\subsection{References} References should be numbered
consecutively throughout the paper and be listed at the end
of the paper with the main heading {\bf References}.
When citing references in the text, type the
corresponding number in square brackets \cite{1}.
References should be complete and in standard IEEE style.
See the section {\bf References} for examples of listing
different kinds of publications \cite{1}--\cite{3}.

\subsection{Equations}
Equations have to be numbered consecutively with the number
in parenthesis, flush to the right. E.g., see
the equation (\ref{E1})
\begin{equation}                        \label{E1}
\sigma(t)=\sigma_0(t)-\int_0^T \gamma(t-s)u(s)\,ds.
\end{equation}

\subsection{Theorems, Lemmas, Remarks, Definitions and so on}
The body of a theorem, or a similar statement, can be typed
either in roman, or in italic font. When using \LaTeX~employ the
\verb|\newtheorem| command. E.g.,
\begin{lemma}
Body of the lemma. Body of the lemma. Body of the lemma.
Body of the lemma. Body of the lemma. Body of the lemma.
Body of the lemma. Body of the lemma. Body of the lemma.
\end{lemma}

For proofs the following style is recommended:

\begin{proof}
Body of the proof. Body of the proof. Body of the proof.
Body of the proof. Body of the proof. Body of the proof.
Body of the proof. Body of the proof. Body of the proof.
\end{proof}
When using LaTeX, the {\bf proof} environment, defined
in the style file {\bf coc2000.sty} can be applied.

\subsection{Illustrations}
Only simple line drawings and graphs are allowed.
Halftone illustrations (pictures) and photographs
are not acceptable. Do not supply Xerox copies.
For the best reproduction and handling of illustrations, we prefer
that they be included in your wordprocessor file or that the
original be pasted.  Many wordprocessors allow the insertion
of figures from a drawing program or the insertion of a
scanned image.  For pasting of original illustrations use glue
such as a glue stick.  Please do not use clear tape.  Do not
use poor photostats, ozalids, blueprints, hectos, Xerox, etc.,
for the drawings, illustrations, or charts, since shades of
blue, green, and brown do not photograph successfully.

Please, consider that poor quality figures
and halftone illustrations usually
require manual retouching,
which increases publication charges considerably.

Figures should be inserted near their citation or at the end of the
manuscript, after References.  Large figures may extend
over two columns if necessary.  Make sure that you number
and include a caption
for each figure. See Fig.~\ref{Fig1} and Fig.~\ref{Fig2}
for examples.
\begin{figure}
\begin{center}
\begin{picture}(220,60)
\put(20,40){\circle{15}}
\put(23,28){{\small $-$}}
\put(15,35){\line(1,1){10}}
\put(25,35){\line(-1,1){10}}
\put(28,40){\vector(1,0){36}}
\put(34,45){$\sigma(t)$}
\put(64,30){\framebox(40,20){PM}}
\put(104,40){\vector(1,0){36}}
\put(109,45){$f(t)$}
\put(140,30){\framebox(40,20){CLP}}
\put(180,40){\line(1,0){32}}
\put(212,40){\line(0,-1){30}}
\put(212,10){\line(-1,0){192}}
\put(20,10){\vector(0,1){23}}
\end{picture}
\end{center}
\caption{One-column sample picture}\label{Fig1}
\end{figure}
\begin{figure*}
\begin{center}
\begin{picture}(300,100)
\put(50,80){\circle{15}}
\put(35,85){{\small $+$}}
\put(53,68){{\small $-$}}
\put(45,75){\line(1,1){10}}
\put(55,75){\line(-1,1){10}}
\put(8,80){\vector(1,0){34}}
\put(8,85){$\psi(t)$}
\put(58,80){\vector(1,0){36}}
\put(64,85){$\sigma(t)$}
\put(94,70){\framebox(40,20){PWM}}
\put(134,80){\vector(1,0){36}}
\put(152,80){\circle*{2}}
\put(138,85){$f(t)$}
\put(178,80){\circle{15}}
\put(163,85){{\small $+$}}
\put(181,68){{\small $-$}}
\put(173,75){\line(1,1){10}}
\put(183,75){\line(-1,1){10}}
\put(185.5,80){\vector(1,0){30}}
\put(216,70){\framebox(40,20){$W(s)$}}
\put(271,85){$\xi(t)$}
\put(256,80){\line(1,0){42}}
\put(298,80){\line(0,-1){70}}
\put(298,10){\line(-1,0){248}}
\put(50,10){\vector(0,1){62}}

\put(216,30){\framebox(40,20){$R(s)$}}
\put(276,80){\circle*{2}}
\put(276,80){\line(0,-1){40}}
\put(276,40){\vector(-1,0){20}}
\put(216,40){\line(-1,0){38}}
\put(190,45){$v(t)$}
\put(178,40){\vector(0,1){10}}
\put(178,50){\circle*{2}}
\put(178,63){\vector(0,1){10}}
\put(178,63){\circle*{2}}
\put(152,80){\line(0,-1){26}}
\put(152,54){\vector(1,0){20.5}}
\thicklines
\put(178,50){\line(-1,1){10}}
\end{picture}
\end{center}
\caption{Two-columns sample picture}\label{Fig2}
\end{figure*}

\subsection{Tables}
Table heading, including the number, is to be typed
above the table. See, e.g.,
\begin{table}[htb]
\begin{center}
\caption{Caption text}
\begin{tabular}{lll}
\hline
Title 1 & Title 2 & Title 3\\
\hline
Row 1, Col 1 & Row 1, Col 2 & Row 1, Col 3\\
Row 2, Col 1 & Row 2, Col 2 & Row 2, Col 3\\
Row 3, Col 1 & Row 3, Col 2 & Row 3, Col 3\\
\hline
\end{tabular}
\end{center}
\end{table}

\subsection{Footnotes}
Please, do not use footnotes.
Authors' addresses have to be
placed after the authors' names below the title of the paper.
Information about any financial support should be placed
in the subsection {\bf Acknowledgements} at the end of your paper.

\subsection{Page Numbers, Session Number,
Copyright Information and Conference Identifications}

Please, do not print page numbers.
All the information listed above
will be inserted later by the proceedings printer.
The name of the author and page number should be written on the
reverse of each sheet, using light pencil.

\section{Headings}

Main headings are to be column centered in a bold font.
They may be numbered, if so desired.

\subsection{Subheadings}
Subheadings should be in a bold font. They should start at
the left-hand margin on a separate line.

\subsubsection{Sub-subheadings} They are to be in a bold font,
should be indented and run in at the beginning of the paragraph.


\section{Recommendations for Preparing Camera-ready Paper
Originals for Printing}

1) Copy should be clean, dark, and readable.
Authors are encouraged to use a laser printer,
300 dpi or higher.

2) Avoid paste ups where possible.
If you must affix an illustration, table, or graphic,
use a glue stick (do not use tape).

3) Do not submit overlays or negatives.
These should be converted to a paper original by you.
Also, copies from a fax machine do not reproduce as well as the original.

4) Do not write side notes or print page numbers on the originals.
Notes and page numbers are to be placed on the back side using pencil.

5) Do not double space your text.

6) Do not fold or bend your originals, if possible,
in mailing or shipping.
If possible, use cardboard stiffeners to protect your
manuscript when mailing to the printer.


\section{Hints for \LaTeX~Users}

It is highly recommended that your manuscript should be
prepared using \LaTeX~document preparation system.
\LaTeX~users have to employ a style file {\bf coc2000.sty}
which is available from COC'2000 home page
{\bf http://www.ipme.ru/coc2000.html},
or can be sent
by e-mail on your request (please apply
{\bf  coc2000@ccs.ipme.ru}).
The \LaTeX~file, containing this instruction, is partly
given in Appendix and can serve as a specimen.

\section{Conclusions}

Please make an extra effort to follow these guidelines
as the quality of the publications depends on you.
Thank you for your cooperation and contribution.
We look forward to seeing you at our conference in
St.Petersburg.

\subsection*{Acknowledgements}

This subsection can be used to acknowledge any kind
of support to your paper (e.g., you can mention your grants).

\begin{thebibliography}{55}
\bibitem{1} A.L.~Fradkov and A.Yu.~Pogromsky,
    {\it Introduction to Control of Oscillation and Chaos},
    World Scientific, Singapore, 1997.
\bibitem{2} A.L.~Fradkov, ``Synthesis of an adaptive system
    of linear plant stabilization,'' {\it Avtomat. i Telemekh.},
    no.~12, pp.~96--103, 1974 (Russian).
\bibitem{3} H.~Nijmeijer, I.I.~Blekhman, A.L.~Fradkov,
    and A.Yu.~Pogromsky,
    ``Self-synchronization and controlled synchronization,"
    {\it Proceedings of the 1st International Conference on
    Control of Oscillations and Chaos
    COC'97}, St.Petersburg, Russia,
    August 27--29 1997, vol.~1, pp.~36--41.
\end{thebibliography}

\section*{Appendix}
Below you find a shortened version of these
instructions written in \LaTeX.
\begin{verbatim}
\documentstyle{coc2000}
\begin{document}
\bibliographystyle{unsrt}
\pagestyle{empty}
\newtheorem{lemma}{Lemma}

\title{Author's Guide\\
2nd International Conference\\
``CONTROL OF OSCILLATIONS AND CHAOS"
(COC'2000)
}

\author{Alexander N. Churilov\\
Dept. of Computer Science,
Marine Technical University\\
3 Lotsmanskaya Str.,
St.Petersburg 190008, Russia\\
churil@gelig.usr.pu.ru}
\date{}
\maketitle

\begin{abstract}
...........................................
\end{abstract}

\section{Manuscript Submission}

Authors submit their manuscripts to the
Organizing Committee.
...........................................

\section{General Specifications}
...........................................

\subsection{References} References should
be numbered consecutively throughout the
paper and be listed at the end of the paper
with the main heading {\bf References}.
When citing references in the text, type
the corresponding number in square brackets
\cite{1}. References should be
complete and in standard IEEE style.
See the section {\bf References} for
examples of listing different kinds of
publications \cite{1}--\cite{3}.

\subsection{Equations}
Equations have to be numbered consecutively
with the number in parenthesis, flush to
the right.
E.g., see the equation (\ref{E1})
\begin{equation}                 \label{E1}
\sigma(t)=\sigma_0(t)-\int_0^T
                       \gamma(t-s)u(s)\,ds.
\end{equation}

\subsection{Theorems, Lemmas, Remarks,
Definitions and so on}
The body of a theorem, or a similar
statement, can be typed either in roman, or
in italic font.
When using \LaTeX~employ
the \verb|\newtheorem| command. E.g.,
\begin{lemma}
Body of the lemma. Body of the lemma.
...........................................
\end{lemma}

For proofs the following style is
recommended:

\begin{proof}
Body of the proof. Body of the proof.
...........................................
\end{proof}
...........................................

\subsection{Illustrations}
...........................................

Make sure that you number and include
a caption for each figure. See
Fig.~\ref{Fig1} and Fig.~\ref{Fig2}
for examples.
\begin{figure}
\begin{center}
\begin{picture}(220,60)
...........................................
\end{picture}
\end{center}
\caption{One-column sample
                      picture}\label{Fig1}
\end{figure}
\begin{figure*}
\begin{center}
\begin{picture}(300,100)
...........................................
\end{picture}
\end{center}
\caption{Two-columns sample
                      picture}\label{Fig2}
\end{figure*}

\subsection{Tables}
Table heading, including the number, is to
be typed above the table. See, e.g.,
\begin{table}[htb]
\begin{center}
\caption{Caption text}
\begin{tabular}{lll}\hline
Title 1 & Title 2 & Title 3\\ \hline
Row1, Col1 & Row1, Col2 & Row1, Col3\\
Row2, Col1 & Row2, Col2 & Row2, Col3\\
Row3, Col1 & Row3, Col2 & Row3, Col3\\
\hline
\end{tabular}
\end{center}
\end{table}
...........................................

\section{Headings}

Main headings are to be column centered
in a bold font.
They may be numbered, if so desired.

\subsection{Subheadings}
Subheadings should be in a bold font. They
should start at the left-hand margin on
a separate line.

\subsubsection{Sub-subheadings} They are
to be in a bold font, should be indented
and run in at the beginning of
the paragraph.

...........................................

\subsection*{Acknowledgements}

This subsection can be used to acknowledge
any kind of support to your paper
(e.g., you can mention your grants).

\begin{thebibliography}{55}
\bibitem{1} A.L.~Fradkov and
A.Yu.~Pogromsky, {\it Introduction to
Control of Oscillation and
Chaos}, World Scientific, Singapore, 1997.
\bibitem{2} A.L.~Fradkov, ``Synthesis of
an adaptive system of linear plant
stabilization,''
{\it Avtomat. i Telemekh.},
no.~12, pp.~96--103, 1974 (Russian).
...........................................
\end{thebibliography}
\end{verbatim}
\end{document}



