\documentstyle{article}

\topmargin=-10mm
\oddsidemargin=-5mm
\evensidemargin=-5mm
\textwidth 16cm
\textheight 23cm
\hfuzz=1.5pt
\makeindex

\font\msbm=msbm10
\font\wr=jwr8
\font\wit=jwti8

\def\al {\alpha}
\def\be {\beta}
\def\br {\hbox{\msbm R}}
\def\de {\delta}
\def\e  {\eta}
\def\ga {\gamma}
\def\f  {\frac}
\def\la {\lambda}
\def\liml{\lim\limits}
\def\om {\omega}
\def\sg {\sigma}
\def\t  {\tau}
\def\ve {\varepsilon}


%\English

\begin{document}
\pagestyle{plain}
\large

\begin{center} {\bf
THE BROCKETT STABILIZATION PROBLEM
}
\end{center}

\centerline{\it Gennadij A. Leonov}

\medskip

\centerline{Dept. of Mathematics and Mechanics}
\centerline{St.Petersburg State University}
\centerline{2, Bibliotechnaya pl., Peterhof}
\centerline{St.Petersburg 198504, Russia}
\centerline{leonov@math.spbu.ru}

\bigskip

\bigskip

In the work [1]  R. Brockett has formulated the following problem.

Given  a triple constant matrices $A,B,C$ under what circumstances does
there exist a time-dependent matrix $K(t)$ such that the  system
$$
\f{dx}{dt}=Ax+BK(t)Cx,\quad x\in\br^n
\eqno(1)
$$
is asymptotically stable.

In [1] it was noted that the problem is still interesting  and unsolved
in the case when  $K(t)$ is a scalar function.

Notice that a stabilization of system (1) with help of a constant
matrix  $K$ is a classical problem  of the control theory.
A reformulation of the Brockett problem from this point of view
is in following.

How much would an application of a time dependent matrix $K(t)$ extend the
capabilities of stabilization?

In a number of cases the answer to this question can be found by means of
the following algorithm for stabilization.

Suppose that there exists a matrix $K_1$ such that a system
$$
\f{dx}{dt}=(A+B\mu K_1C)x
\eqno(2)
$$
with a scalar parameter $\mu$  possesses a stable linear invariant manifold
$L(\mu)$ for $\mu\geq\mu_0$. Here
$\mu_0$ is some number.

We also assume that
$$
\liml_{\mu\to+\infty}L(\mu)=L_0
\eqno(3)
$$
and for any number  $\de>0$ there exists $\mu\geq\mu_0$ such that
$$
|x(1,x_0)|\leq\de,\quad\forall\,x_0\in\{|x_0|=1\}\cap L(\mu).
\eqno(4)
$$
Here $x(0,x_0)=x_0$ and a limit (3) is understood in the sense that
$L(\mu)\cap\{|x|\leq1\}$ is situated in a $\ve$-neighborhood of
$L_0\cap\{|x|\leq1\}$, where $\ve\to0$ as  $\mu\to+\infty$.

The assumption we have formulated means a fast convergence of trajectories
on manifold  $L(\mu)$ for large parameter $\mu$.

Let us denote by $M(\mu)$ a linear invariant manifold of system (2),
for which we have
$$
\liml_{\mu\to+\infty}M(\mu)=M_0,
$$
$$
\dim M(\mu)+\dim L(\mu)=n,\quad M(\mu)\cap L(\mu)=\{0\}.
$$

Suppose that $M(\mu)$ is a manifold of slow movements,
i.e., there exists a number $R_1$ such that for all $\mu\geq\mu_0$ the
inequality holds
$$
|x(1,x_0)|\leq R_1,\quad\forall\,x_0\in\{|x_0|=1\}\cap M(\mu).
\eqno(5)
$$

Now assume that there exists a matrix $K_2$ such that for the system
$$
\f{dy}{dt}=(A+BK_2C)y
\eqno(6)
$$
a number $\t$ exists such that
$$
Y(\t)M_0\subset L_0.
\eqno(7)
$$
Here $Y(t)$ is a fundamental matrix of system (6), $Y(0)=I$.

Let us denote a $(2+\t)$-periodic matrix $K(t)$ in the following way
$$
K(t)=\left\{\begin{array}{ll}
\mu K_1 &\hbox{for}\quad t\in[0,1),\cr\cr
K_2 &\hbox{for}\quad t\in[1,1+\t),\cr\cr
\mu K_1 &\hbox{for}\quad t\in[1+\t,2+\t),
\end{array}\right. \eqno(8)
$$

\smallskip

{\bf Theorem 1.} {\it
A system {\rm(1)} with a matrix $K(t)$ of the type {\rm(8)} for sufficiently
large $\mu$ is asymptotically stable.}

\medskip

{\bf Proof.}
The construction of manifolds $L_0$ and $M_0$ implies that for any
number $\ve>0$ there exists $\mu_1\geq\mu_0$ satisfying the following
property.

For $t=1$ the solution $x(t,x_0)$ of system (1) with $\mu\geq\mu_1$
and with the initial data
$x_0=x(0,x_0)$, $|x_0|=1$ is situated in a $\ve$-neighborhood of manifold
$$
M_0\cap\{|x|\leq R_1\}.
$$
It is clear that there exists a number $R_2$ having the following property.
For $t=1+\t$ the solution $x(t,x_0)$
with the initial data $x_0=x(0,x_0)$, $|x_0|=1$
is situated in an $R_2\ve$-neighborhood of manifold
$$
L_0\cap\{|x|\leq R_2\}.
$$
It can readily be seen that for $t=2+\t$ for the solution considered
the following estimate
$$
|x(2+\t,x_0)|\leq\nu\ve
$$
is valid if $\mu$ is sufficiently large number, $\nu$ is some
number.

Choosing $\ve$ sufficiently small (and therefore $\mu$ sufficiently  large)
we obtain that for all $x_0$ from a sphere $|x_0|=1$ the following estimate
$$
|x(2+\t,x_0)|<\f12\,
$$
is valid.
The latter means an asymptotical stability  of system (1)
with a periodical matrix $K(t)$ of the type (8).

Theorem 1 implies some corollaries.

Let us consider a periodic solution $z(t)$ of system
$$
\dot z=Qz,\quad z\in\br^n
$$
where $Q$ is an $(n\times n)$-matrix such that $\det Q\neq0$.

\medskip

{\bf Lemma.} {\it
For any vector $h\in\br^n$ there exists a number $\tau$ such
that $h^*z(\tau)=0$.}

\medskip

{\bf Proof.}
Let $h^*z(t)>0$ $\ \forall\,t\in\br^n$. Then the periodicity of
function $h^*z(t)$ implies the following relation
$$
\lim\limits_{t\to+\infty}\int\limits_0^t h^*z(t)\,dt=+\infty.
$$
On the other hand, we have
$$
\int\limits_0^t h^*z(t)\,dt=h^*Q^{-1}\big(z(t)-z(0)\big).
$$
Whence it follows that the integral is a bounded function for
$t\geq0$. This contradiction proves the existence of a number
$\tau$ such that $h^*z(\tau)=0$.

\medskip

{\bf Theorem 2.} {\it
Let there exist matrices $K_1$ and $K_2$ satisfying the following conditions:

$1)$ a matrix $BK_1C$ has $n-1$ eigenvalues with negative real parts and
$\det BK_1C=0$,

$2)$ for a vector $u\neq0$, satisfying an equality $BK_1Cu=0$, and for
a certain number $\la$, the vector-function
$$
\exp\big[(A+BK_2C+\la I)t\big]u
\eqno(9)
$$
is periodic.

Then there exists a periodic matrix $K(t)$ such that system $(1)$
is asymptotically stable.
}

\medskip

To prove this theorem it is necessary to check a satisfaction of
properties (3)--(7).

>From hypothesis 1) of Theorem 2 the properties (3)--(5) follow. Here
hyperplane $L_0$ is an $(n-1)$-dimensional stable manifold of
the following linear system
$$
\f{dy}{dt}=BK_1Cy.
$$

>From hypothesis 2) of Theorem 2 and the lemma it follows that
there exists $\tau$ such that
$$
\exp\big[(A+BK_2C)\tau\big]u\in L_0.
\eqno(10)
$$

Thus, we can see that by Theorem 1 system (1) with a matrix of
the type (8) is asymptotically stable.

\medskip

{\bf Theorem 3.} {\it
Let $n=2$ and there exist matrices $K_1$, $K_2$, and number $\la$ such that
the following conditions hold:

$1)$ $\det BK_1C=0$ and $\hbox{\rm Tr}\,BK_1C\neq0$,

$2)$ a matrix $A+BK_2C+\la I$ has purely imaginary eigenvalues.

Then there exists a periodic matrix $K(t)$ such that system $(1)$
is asymptotically stable.
}

\medskip

 In order to prove this it should be noted that  hypothesis 1) of Theorem 3
implies the existence
of a nonzero eigenvalue of matrix $BK_1C$. If this eigenvalue is negative, then
the hypothesis 1) of Theorem 2 is satisfied. If it is positive, then
the hypothesis 1) of Theorem 2 is satisfied for the matrix $-K_1$.

>From the hypothesis 2) of Theorem 3 it follows that
the hypothesis 2) of Theorem 2 is satisfied.
Theorem 3 is proved.

We now consider a case that $n=2$, $\ B$ is a column vector and $C$ is a
row vector. In this case the transfer function  $W(p)$ of system (1)
takes the form
$$
W(p)=C(A-pI)^{-1}B=\f{\rho p+\ga}{p^2+\al p+\be}\,.
$$

Let $\rho\neq0$. Then, we can assume without loss of generality
that $\rho=1$. Let us also suppose that the inequality holds
$$
\ga^2-\al\ga+\be\neq0.
$$
This condition assures a nondegeneracy of transfer function  $W(p)$.
It is well known [2] that in this case system (1) can be reduced to
the  following form
$$
\begin{array}{l}
\dot\sg=\e,\cr\cr
\dot\e=-\al\e-\be\sg-K(t)(\e+\ga\sg).
\end{array} \eqno(11)
$$
Here $K(t)$ is a scalar function.

The problem on a stabilization with help of
a constant value $K(t)\equiv K_0$ has a positive solution if and
only if
$$
\al+K_0>0,\quad \be+\ga K_0>0.
$$
For the existence of a number $K_0$, satisfying these inequalities,
it is necessary and sufficient that the inequalities
either $\ga>0$ or $\ga\leq0$, $\ \al\ga<\be$ where satisfied.

Now we consider the case that a stabilization with help of a constant
$K_0$ is impossible:
$$
\ga\leq0,\quad \al\ga>\be.
$$

Let us check the satisfaction of the hypotheses of Theorem 3. It can easily be
seen that hypothesis 1) of Theorem 3 is satisfied.
For a satisfaction of the hypothesis 2) of Theorem 3
it is necessary and sufficient that for some numbers
$k_2$ and $\la$ a polynomial
$$
p^2+(\al+k_2-2\la)p+(\be+k_2\ga-\la^2)
$$
has a purely imaginary zeroes. This is valid in the case that
$$
\al+k_2-2\la=0,\quad\be+k_2\ga-\la^2>0.
\eqno(12)
$$
For the existence of $\la$ and $k_2$ both, satisfying condition (12),
it is necessary and sufficient that the following inequality holds
$$
\ga^2-\al\ga+\be>0.
\eqno(13)
$$

Thus, if inequality (13) is valid, then by Theorem 3
there exists a periodic function $K(t)$ such that system (11)
is asymptotically stable.

This implies that if an inequality
$$
\ga^2-\al\ga+\be<0
\eqno(14)
$$
is valid, then the estimate
$$
(\e+\ga\sg)^\bullet=(-\ga^2-\be+\al\ga)\sg>0,
$$
$\forall\,\sg>0$, $\e=-\ga\sg$ is true. Whence it follows that for $\sg(0)>0$,
$\ \e(0)+\ga\sg(0)>0$ the inequality holds
$$
\e(t)+\ga\sg(t)>0,\quad\forall\,t\geq0.
$$
But then $\dot\sg(t)=\e(t)>0$, $\forall\,t\geq0$. Hence the instability
of system (11) follows.

Thus, we have proved the following result.

\medskip

{\bf Theorem 4.} [3] {\it
If inequality $(13)$ holds, then system $(11)$
is stabilized with help of a periodic function $K(t)$.

If inequality $(14)$ holds, then  there do not exist functions $K(t)$,
for which system $(11)$ is asymptotically stable.
}

\medskip

This theorem was proofed also by another class of functions
$K(t)$:
$$
K(t)=(K_0+K_1\om\cos\om t),\quad \om\gg1
$$
and averaging methods [4].

Let us consider system (1) with  $n=3$, $C$ is row vector, $B$ is
column vector, $K(t)$ is function: $\br^1\to\br^1$.

{\bf Theorem 5.} {\it
Let $CB\neq0$ and there exist a numbers $K_1,K_2$ satisfying the
following conditions:

$1)$ a matrix $A+K_1BC$ has two complex eigenvalues and one
negative eigenvalue,

$2)$ function
$$
C\exp[(A+K_2BC)t]B
$$
has at least one zero on interval $(-\infty,0)$.

Then there exists a periodic function $K(t)$ such that system
$(1)$ is asymptotically stable.
}

\bigskip

\medskip
\noindent
{\bf References}

\begin{itemize}
\item[{[1]}]
Brockett R.
A stabilization problem.
Open Problems in Mathematical Systems and Control Theory.
Springer, 1999.

\item[{[2]}]
Lefschetz S.
Stability of Nonlinear Control Systems.
Academic Press. New York. 1965.

\item[{[3]}]
Leonov G.A.
The Borckett Stabilization Problem.
Proceedings of International Conference. Control of Oscillations
and Chaos. St. Petersburg, Russia, July, 2000. P.38--39.

\item[{[4]}]
Morean L., Aeyels D.
Stabilization by means of periodic output feedback. Proceedings
of Conference of Decision and Control. Phoenix, Arizona USA,
Desember 1999. P.108--109.
\end{itemize}

\end{document}

