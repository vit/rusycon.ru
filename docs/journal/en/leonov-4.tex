\documentstyle{article}

\topmargin=-10mm
\oddsidemargin=-5mm
\evensidemargin=-5mm
\textwidth 16cm
\textheight 23cm
\hfuzz=1.5pt
\makeindex

\font\msbm=msbm10 scaled 1200
\font\wr=jwr8 scaled 1200
\font\wit=jwti8 scaled 1200

\def\al {\alpha}
\def\ba{\begin{array}}
\def\be {\beta}
\def\br {\hbox{\msbm R}}
\def\de {\delta}
\def\ea{\end{array}}
\def\e  {\eta}
\def\ga {\gamma}
\def\f  {\frac}
\def\la {\lambda}
\def\liml{\lim\limits}
\def\om {\omega}
\def\sg {\sigma}
\def\t  {\tau}
\def\ve {\varepsilon}


\begin{document}
\pagestyle{plain}
\large

\begin{center}
THE LYAPUNOV FUNCTIONS IN ESTIMATES OF A DIMENSION OF DYNAMICAL
SYSTEM ATTRACTORS


\medskip

{\it G.A. Leonov}

\smallskip

{Dept. of Mathematics and Mechanics}

{St.Petersburg State University}

{2, Bibliotechnaya pl., Peterhof}

{St.Petersburg 198504, Russia}

{leonov@math.spbu.ru}
\end{center}

\medskip

{\wit
Abstract:
}

\bigskip

\medskip

The present work was reported in the Rokhlin memorial conference in August
1999. It incorporates a review of the results, concerning  estimations of
attractor dimensions by the Lyapunov functions.

This line of investigation made possible to obtain the highly simple formulas
for computing the Lyapunov dimension of the Henon and Lorenz attractors

Consider the Henon mapping $F:\br^2\to\br^2$
$$
\ba{l}
z\to a+by-z^2,\cr
y\to z,
\ea
$$
where $a>0$, $b\in(0,1)$.
We shall consider here the Lyapunov dimension $\dim_L\,K$ of an invariant
set $K$: $FK=K$, containing the following stationary points
$$
z_{\pm}=\f12\,\left[b-1\pm\sqrt{(b-1)^2+4a}\,\right].
$$

{\bf Theorem 1.} {\it
For the mapping $F$ the equality
$$
\dim_L\,K=1+\f1{1-\ln b/\ln\al(z_-)}\,,
$$
where $\al(z_-)=\sqrt{z_-^2+b}-z_-$,
}
is valid.
\medskip

We now consider the Lorenz system
$$
\ba{l}
\dot x=-\sg x+\sg y,\cr
\dot y=rx-y-xz,\cr
\dot z=-bz+xy,
\ea \eqno(1)
$$
where $r,b,\sg$ are positive numbers. Suppose that
$$
\ba{l}
b\geq2,\qquad \sg+1\geq b,\cr
r\sg^2(4-b)+2\sg(b-1)(2\sg-3b)>b(b-1)^2
\ea
$$
and denote by  $K$ an invariant set of system (1),
containing the point $x=y=z=0$.

\medskip

{\bf Theorem 2.} {\it
The following equality holds
$$
\dim_L\,K=3-\f{2(\sg+b+1)}{\sg+1+\sqrt{(\sg-1)^2+4r\sg}}\,.
$$
}


\end{document}

